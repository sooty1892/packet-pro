\documentclass[interim_report.tex]{subfiles}

\begin{document}

\section{Evaluation Plan}
The sections below outline how the project will be evaluated in order to determine whether the initial objectives have been met and whether the final outcome can be deemed a success, even if it provides unexpected results.

\subsection{Experiments \& Outcomes}
The main measure of success will be from the experiments which are to be carried out. Using the high performance Java packet I/O application which is currently been developed as part of this project, it will be compared to similar, already existing applications written in pure Java and in C/C++. This comparison will consist of 2 parts, firstly an application for a NAT and secondly a simple IP address firewall. These similar applications will then be benchmarked against each other and the results analysed. \\
\newline
Benchmarking will consist of setting up 2 independent computers, most likely standalone machines, in order to take advantage of the ability to overwrite the Intel network card drivers as required by the DPDK API. Testing can the be carried out on the firewall and NAT implemented in the 3 different contexts, measuring the latency between the time sent and time received of the network packet. As the testing will be carried out on the same machines, linked to the same network there is unlikely to be much variation in the network latency. This means that the variation in the times will be from the system processing speeds. To account for minor variations in the computing performance of the system, numerous iterations of the same test will be carried out, then taking statistical averages will provide the best final results in order to analyse correctly. \\
\newline
Analysis of the results will be mainly carried out via the use of graphs which allow for easy comparison of results. The expected outcome is that the application developed from within this project is of similar speeds to applications coded directly in C/C++ and much faster than those coded in Java. This is mainly because the kernel will be bypassed, but the use of the Java Native Interface is expected to slow down certain aspects of the application. However, as long as speeds which coincide with those of the line rate are reached, this will be acceptable. There is the possibility that expected speeds are much faster than those that are actually measured. Although this isn't ideal, the project shouldn't be considered a failure as it will still provide very useful feedback as to where future investigations should be focussed on.

\subsection{Functionality \& Usability}
Although the objectives didn't specify any requirement to make the final application highly usable, my personal preference would be to have a product that could be used by anybody in the future. This obviously requires well commented and documented code. As the previous evaluation was all quantitative measurements, this evaluation will consist of qualitative measurements mainly undertaken by myself. \\
\newline
The functionality and usability can also be checked by the project supervisor \& co-supervisor to check if the original specifications were met. This can allow for an appraisal of the final application to be carried out which will be a good indication of the success of the project.

\end{document}